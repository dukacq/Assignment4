\documentclass[10pt,a4paper]{article}
\usepackage{amsfonts}
\usepackage{graphicx}
\usepackage[width=5.00cm, left=2.00cm, right=2.00cm, top=2.00cm]{geometry}
\begin{document}
\textbf{}\\


\begin{center}
\includegraphics[scale=0.8]{../../Pictures/Capture.PNG}\\\vspace{3cm}
  {\Huge \textbf{LITERATURE REVIEW}}
  \\\vspace{1.00cm}{\Large An investigation into the usefulness of the Android Wear Interface}
  \vspace{4cm}
  \\{\LARGE Acquila A. Kakaire}
  \vspace{1.5cm}
  \\{\Large B.Sc Computer Science}
  \vspace{1.5cm}
  \\{\Large 14/U/6840/EVE}
  \date{\today}
  \vspace{7.5cm}
		
\end{center}

\section{Abstract:} 
The aim of this literature review is to investigate and present the smart watch as a potentially useful device for information access used by university students.  In this review the smart watch interface, which is part of wearable computing, is looked at as an alternative data access platform to the already existing and non-wearable smart phone.  Current application usage and development on the smart phone and smart watch interfaces, as well as the information which university students would want from such devices are discussed.  Hardware and software advancements on the smart watch which could be driving access to user information are also covered.

\section{Main Body:}
With the introduction of the smart watch interface to the family of wearable computing devices, the  question  has  been  posed  as  to  whether  smart  watches  can  provide  their  users  with  useful applications.  Smart watches are now seen to be socially acceptable in the modern digital world, and can possibly be used as an alternative interface for information access \cite{Bieber}. Smart watches which were previously available on the market offered a limited number of applications at the discretion of manufacturers, with no intention of taking advantage of any open source software solutions \cite{Smith}.  The standardization of Application Programming Interfaces (API’s) and the increasing availability of libraries through Software Development Kits (SDK’s)  for  these  devices, means  that  new  applications  can  be  created  relatively  quickly  by developers and are able to meet the needs of users more dynamically than in the past \cite{Sachse}.

Applications for smart watches are being developed on open platforms such as Android which has been seen with Sony’s \textit{Open  Smartwatch} project \cite{Sony} and Pebble’s open development project \cite{Pebble}.  As seen by the findings of Narayanaswami and Raghunath \cite{2000} as well as Hutterer, Smith, Thomas, Piekarski, and Ankcorn \cite{2005} there have already been investigations into the need to use such a device over already established platforms and what data the users of these devices would want.  Further studies regarding the usability of smart watches have also been conducted by Pascoe and Thomson \cite{2007}.  This literature review will investigate the smart watch as a possible alternative interface to the smart phone for the access of data desired by university students.

\section{Summary:}
This  literature  review  has  discussed  the  smart  watch  interface  within  the  domain  of wearable computing with specific focus on the \textit{Pebble}. The \textit{Pebble} has been compared to with other smart watches  currently  available  such  as  the Samsung  \textit{Galaxy  Gear  2, Sony  SmartWatch  2, Kreyos Meteor, Neptune Pine, Martian} and the \textit{Agent}. The limited disadvantages on the \textit{Pebble} were also discussed, such as the minimal display, efficient power management and simple user interaction techniques used. The smart watch has been analyzed as an alternative interface to the conventional smart phone, and the differences between these interfaces have been addressed.  Convenience and always accessible data access have been found to be the driving factors for the popularity of the smart watch interface amongst users.

\begin{thebibliography}{8}
  
\bibitem{Bieber} Bieber, G., Haescher, M., \& Vahl, M.  (2013).  \textit{Sensor Requirements for Activity Recognition on Smart Watches.}  In Proceedings of the 6th International Conference on Pervasive Technologies Related to Assistive Environments (pp. 67:1–67:6).  New York, NY, USA: ACM.

\bibitem{Smith} Smith, N.  (2013, December).  Classic Project:  Pulsar P1 \& P2 Quartz Wristwatch. Engineering \& Technology, 8, 100-101(1).

\bibitem{Sachse} Sachse, J.  (2010). The standardization of Widget-APIs as an approach for overcoming device fragmentation (Unpublished master’s thesis).  HTW-Berlin (University of Applied Sciences)

\bibitem{Sony} Sony.  (2014a). Open smartwatch project. Retrieved from http:/developer.sonymobile.com/services/open-smartwatch-project/ (Accessed on 27/05/2014

\bibitem{Pebble} Pebble.  (2014). Develop for \textit{pebble}. Retrieved from https://developer.getpebble.com/ (Accessed on 27/05/2014)

\bibitem{2000} Narayanaswami, C., \& Raghunath, M.  (2000).  Application Design for a Smart Watch with a High Resolution Display. 2012 16th International Symposium on Wearable Computers, 0, 7.

\bibitem{2005}Hutterer, P., Smith, M. T., Thomas, B. H., Piekarski, W., \& Ankcorn, J.  (2005).  Lightweight User Interfaces for Watch Based Displays.  In Proceedings of the Sixth Australasian Conference on User Interface - Volume 40 (pp. 89–98).  Darlinghurst, Australia, Australia: Australian Computer Society, Inc.

\bibitem{2007}Pascoe, J., \& Thomson, K.  (2007).  On the Use of Mobile Tools in Everyday Life.  In Proceedings of the 19th Australasian Conference on Computer-Human Interaction:  Entertaining User Interfaces (pp. 39–47).  New York, NY, USA: ACM.

\end{thebibliography}

\end{document}